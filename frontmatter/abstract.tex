%
% File: abstract.tex
% Author: V?ctor Bre?a-Medina
% Description: Contains the text for thesis abstract
%
% UoB guidelines:
%
% Each copy must include an abstract or summary of the dissertation in not
% more than 300 words, on one side of A4, which should be single-spaced in a
% font size in the range 10 to 12. If the dissertation is in a language other
% than English, an abstract in that language and an abstract in English must
% be included.

\chapter*{Abstract}
\begin{SingleSpace}
\initial{M}arkov models  are a popular technique for understanding the dynamics of systems which move through ``rough'' potentials \cite{zwanzigDiffusionRoughPotential1988}. In such cases, the system is well approximated as transitioning between discrete states with a set state-to-state probability, independent of its history. Choosing how these states relate to the coordinates of the system (the discretization) and how these are partitioned into metastable sets (the coarse graining) is of central importance to the technique.  This thesis contributes to methods for making these choices and applies them to two systems: water diffusion and enzyme dynamics.
 
Markov models were used to provide an explanation of water diffusion through viscous aerosol particles, where diffusion is known to diverge from typical Stokes-Einstein behaviour. The choice of discretization and coarse-graining techniques came from established methods and heuristics in the Markov modelling literature. The analysis showed that water diffuses by hopping between transient cavities created by the organic fraction of the aerosol particle. For the majority of the time this process is irreversible but the water can also establish local equilibria between clusters of cavities arresting the diffusion process. 

A more complex workflow was proposed and evaluated for the case of the aromatic amine dehydrogenase, an enzyme at the heart of the debate surrounding hydrogen tunneling and enzyme dynamics.  This workflow used ideas from the statistics and machine learning communities in order to make the modelling  process more transparent, efficient and reproducible.  The response surface of an MSM - the change in model quality in response to modelling choices - was estimated and optimised using Bayesian optimisation. Statistical model selection techniques for selecting the number of metastable states in a hidden Markov model were evaluated. Theoretical and practical arguments are made in favour of the integrated complete-data likelihood criterion. The benefits of this more elaborate workflow were mixed.  The response surface proved useful in creating tests of the sensitivity of inferences to the modelling choices.  Many of the modelling choices were shown to not affect the model quality and as a result Bayesian optimisation proved of little benefit. The conformational landscape of aromatic amine dehydrogenase was found to consist of many short lived  (\SIrange{20}{300}{\nano\second}) metastable states which slowly interconvert on a timescale of approximately \SI{1.2}{\micro\second}. However, the simulations had moved away from their reactive conformations and so the implications for understanding reactivity were limited. In addition, these results could not be validated and sensitivity tests cast doubt on the robustness of this conclusion. The source of these problems was investigated and several solutions were proposed.  


 
% In contrast, the approach for constructing MMs for large biomolecules typically uses a complex processing pipeline, involving many modelling choices (hyperparameters) to create discrete representations of the dynamics. This thesis demonstrates Bayesian hyperparameter optimisation and response surface methods, common within the machine learning community, to both choose hyperparameters in an efficient and reproducible way, and to describe how these hyperparameters affect model quality. Statistical model selection techniques for selecting the number of metastable states in hidden Markov model coarse-grained representations are evaluated. Theoretical and practical arguments are made in favour of the integrated complete-data likelihood criterion (ICL). A simulation of AADH, an enzyme at the heart of the debate on the role of dynamics in catalysis and tunnelling, was created and critically evaluated. Bayesian optimised discretizations and ICL selected HMMs are used to describe the conformational landscape of AADH and the relevance for understanding its kinetics are discussed.
 

\end{SingleSpace}
\clearpage